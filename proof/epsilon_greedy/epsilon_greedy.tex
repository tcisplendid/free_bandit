%
% CSE Electronic Homework Template
% Last modified 8/23/2018 by Jeremy Buhler

% \documentclass[11pt]{article}
\documentclass[fleqn]{article}

\usepackage[left=0.7in,right=0.7in,top=1in,bottom=0.7in]{geometry}
\usepackage{fancyhdr} % for header
\usepackage{graphicx} % for figures
\usepackage{amsmath}  % for extended math markup
\usepackage{amssymb}
\usepackage[bookmarks=false]{hyperref} % for URL embedding
\usepackage[noend]{algpseudocode} % for pseudocode
\usepackage[plain]{algorithm} % float environment for algorithms

%%%%%%%%%%%%%%%%%%%%%%%%%%%%%%%%%%%%%%%%%%%%%%%%%%%%%%%%%%%%%%%%%%%%%%
% STUDENT: modify the following fields to reflect your
% name/ID, the current homework, and the current problem number

% Example: 
%\newcommand{\StudentName}{Jeremy Buhler}
%\newcommand{\StudentID{123456}

\newcommand{\StudentName}{Tiancheng He}
\newcommand{\StudentID}{467513}
\newcommand{\HomeworkNumber}{2}
\newcommand{\WustlKey}{tiancheng}

%%%%%%%%%%%%%%%%%%%%%%%%%%%%%%%%%%%%%%%%%%%%%%%%%%%%%%%%%%%%%%%%%%%%%%%%
% You can pretty much leave the stuff up to the next line of %%'s alone.

% create header and footer for every page
\pagestyle{fancy}
\fancyhf{}
\lhead{\textbf{\StudentName}}
\chead{\textbf{\StudentID} }
\rhead{\textbf{wustlkey: \WustlKey}}
\cfoot{\thepage}

% preferred pseudocode style
\algrenewcommand{\algorithmicprocedure}{}
\algrenewcommand{\algorithmicthen}{}

% ``do { ... } while (cond)''
\algdef{SE}[DOWHILE]{Do}{doWhile}{\algorithmicdo}[1]{\algorithmicwhile\ #1}%

% ``for (x in y ... z)''
\newcommand{\ForRange}[3]{\For{#1 \textbf{in} #2 \ \ldots \ #3}}

% these are common math formatting commands that aren't defined by default
\newcommand{\union}{\cup}
\newcommand{\isect}{\cap}
\newcommand{\ceil}[1]{\ensuremath \left\lceil #1 \right\rceil}
\newcommand{\floor}[1]{\ensuremath \left\lfloor #1 \right\rfloor}


%amsfonts give blackboard bold:
\def\AA{{\mathbb A}}\def\BB{{\mathbb B}}\def\CC{{\mathbb C}}
\def\DD{{\mathbb D}}\def\EE{{\mathbb E}}\def\FF{{\mathbb F}}
\def\GG{{\mathbb G}}\def\HH{{\mathbb H}}\def\II{{\mathbb I}}
\def\JJ{{\mathbb J}}\def\KK{{\mathbb K}}\def\LL{{\mathbb L}}
\def\MM{{\mathbb M}}\def\NN{{\mathbb N}}\def\OO{{\mathbb O}}
\def\PP{{\mathbb P}}\def\QQ{{\mathbb Q}}\def\RR{{\mathbb R}}
\def\SS{{\mathbb S}}\def\TT{{\mathbb T}}\def\UU{{\mathbb U}}
\def\VV{{\mathbb V}}\def\WW{{\mathbb W}}\def\XX{{\mathbb X}}
\def\YY{{\mathbb Y}}\def\ZZ{{\mathbb Z}}
%%%%%%%%%%%%%%%%%%%%%%%%%%%%%%%%%%%%%%%%%%%%%%%%%%%%%%%%%%%%%%%%%%%%%%

\begin{document}

% STUDENT: Your text goes here!

1. Definition: $\hat{\mu}_i^m$ means the estimated mean of the Gaussian random variable $i$ after sampling $m$ times.
\medskip

Theorem: Given 2 Gaussian r.v. with the mean $\mu_1, \mu_2$, and $\mu_1 > \mu_2$. $\PP(\hat{\mu}_1^{m_1}>\hat{\mu}_2^{n_1}) \geq \PP(\hat{\mu}_1^{m_2}>\hat{\mu}_2^{n_2})$ if $m_1 \geq m_2, n_1\geq n_2$. They are equal only if $m_1 = m_2, n_1 = n_2$

\medskip

Setting:

Best arm $a_1\sim\mathcal{N}(\mu_1, \sigma^2)$. Pulled $m$ times.

Second best arm $a_2\sim\mathcal{N}(\mu_2, \sigma^2)$. Pulled $n$ times.
\medskip

So,

Estimation $\hat{\mu_1}\sim\mathcal{N}(\mu_1, \frac{\sigma^2}{m})$.

Estimation $\hat{\mu_2}\sim\mathcal{N}(\mu_2, \frac{\sigma^2}{n})$
\medskip

Set $\Delta=\mu_1-\mu_2>0$, 

$s^2=\frac{1}{m}+\frac{1}{n}$,

$X=\hat{\mu_1}-\hat{\mu_2}\sim\mathcal{N}(\mu_1-\mu_2, \frac{\sigma^2}{m}+\frac{\sigma^2}{n})=\mathcal{N}(\Delta, (\sigma s)^2)$,

$\Phi(x)$ as the CDF for standard Gaussian distribution.
\medskip

So, 

\begin{align*}
\PP(\hat{\mu_1}>\hat{\mu_2}) &= \PP(X>0)\\
 &= 1-\PP(X\le 0)\\
 &= 1-\PP(\frac{X-\Delta}{\sigma s}\le -\frac{\Delta}{\sigma s})\hspace{1cm}\text{where $\frac{X-\Delta}{\sigma s}\sim \mathcal{N}(0,1)$}\\
 &=1-\Phi(-\frac{\Delta}{\sigma s})
\end{align*}

\begin{align*} 
n \nearrow \Rightarrow s \searrow \Rightarrow -\frac{\Delta}{\sigma s} \searrow \Rightarrow \Phi(-\frac{\Delta}{\sigma s}) \searrow \Rightarrow \PP(\hat{\mu_1}>\hat{\mu_2}) \nearrow
\end{align*} 

When $n$ increases, $\PP(\hat{\mu_1}>\hat{\mu_2})$ increases monotonously.  
\bigskip

2. Thought
\medskip

Although more singals increase the probability to choose the best arm at certain round, the events of $\hat{\mu_1}<\hat{\mu_2}$ are more serious. It's harder to recover just by pulling $a_2$ more.

Instead we should consider how long it will take to achieve $\hat{\mu_1}>\hat{\mu_2}$. For example, given $\hat{\mu_1}$ and $\hat{\mu_2}$. Denote $Y_n(\hat{\mu_1})$ as the estimation of $a_2$'s mean after $n$ rounds. The rounds it takes to recover might be 

% \begin{equation}
% \EE[\hat{\mu_1}>\hat{\mu_2} \ \text{after $n$ rounds}] = \mathop{\Sigma}_{n=1}^{\infty}n\cdot\PP(\hat{\mu_1}>Y_n, \hat{\mu_1}<\mathop{\min}_{i<n}(Y_i))
% \end{equation}
\bigskip

3. Formulation attempt
\medskip

Set estimation of $\hat{\mu}_1$ $\hat{\mu}_2$ after $t$ rounds as $\hat{\mu}_1^t$ and $\hat{\mu}_2^t$. They are pulled $m^t$ and $n^t$ times.

So, 

\begin{align*}
\PP(\hat{\mu}_1^t>\hat{\mu}_2^t) &= \PP(\hat{\mu}_1^t>\hat{\mu}_2^t|\hat{\mu}_1^{t-1}>\hat{\mu}_2^{t-1})\PP(\hat{\mu}_1^{t-1}>\hat{\mu}_2^{t-1})+\PP(\hat{\mu}_1^t>\hat{\mu}_2^t|\hat{\mu}_1^{t-1}\leq\hat{\mu}_2^{t-1})\PP(\hat{\mu}_1^{t-1}\leq\hat{\mu}_2^{t-1})
\end{align*}

If $\hat{\mu}_1^{t-1}>\hat{\mu}_2^{t-1}$, arm $a_1$ is chosen last round. Then $m=m^{t-1}+1$ in computing $\PP(\hat{\mu}_1^t>\hat{\mu}_2^t|\hat{\mu}_1^{t-1}>\hat{\mu}_2^{t-1})$. Similarly, $n=n^{t-1}+1$ in computing the other condition.

For easier computation, $m^{t}=\EE(m^{t})=m^{t-1}+\PP(\hat{\mu}_1^{t}>\hat{\mu}_2^{t})$
\clearpage





$\hat{\mu}_{1,i}$ is the estimation of $\mu_1$ at round $i$.
\begin{align*}
R(T)&=(\mu_1-\mu_2)\EE[\text{times of choosing arm 2}]\\
&=(\mu_1-\mu_2)\Sigma_{i=1}^T{\PP(\hat{\mu}_{1,i}<\hat{\mu}_{2,i})}
\end{align*}

$\hat{\mu}_{1}^m$ is the estimation of $\mu_1$ after pulling $m$ times.

\begin{align*}
    \PP(\hat{\mu}_1<\hat{\mu}_2) &= \Sigma_{m+n=t}\PP(\hat{\mu}_1^m<\hat{\mu}_2^n|m,n)\PP(m,n)
\end{align*}

$m+n=t$
\begin{align*}
    \PP(m,n) &= \PP(m-1, n)(1-\PP(\hat{\mu}_1^{t-1}<\hat{\mu}_2^{t-1})) + \PP(m, n-1)\PP(\hat{\mu}_1^{t-1}<\hat{\mu}_2^{t-1})
\end{align*}
\bigskip

Denote the sequence of $T$ signals after initialization as $X_1X_2...X_T$. Each $X$ is a random variable that could be $A$(arm1) or $B$(arm2). For example, for a sequence of $A_1B_1B_2A_2$,  $A_i$ means the $i$-th received reward of arm1. $B_i$ means of arm2. $A_1B_1B_2A_2$ means a pulling history of arm1, arm2, arm2, arm1. $A_0$ and $B_0$ are signals received from initialization of algorithms.
\begin{align*}
    \EE[\text{times of choosing arm 2}]=\Sigma_{X_1X_2...X_T}{\PP(X_1X_2...X_T)\cdot(\text{number of }X=B)}
\end{align*}

For example, when round is 2:
\begin{align*}
    \EE[\text{times of choosing arm 2}]&=\Sigma_{X_1X_2...X_T}{\PP(X_1X_2...X_T)\cdot(\text{number of }X=B)}\\
    &= \PP(A_1B_1) + \PP(B_1A_1) + \PP(B_1B_2)*2\\
    &= \PP(A_0>B_0)\PP(\frac{A_0+A_1}{2}<B_0|A_0>B_0)\\
    &+\PP(A_0<B_0)\PP(A_0>\frac{B_0+B_1}{2}|A_0<B_0)\\
    &+2\cdot\PP(A_0<B_0)\PP(A_0<\frac{B_0+B_1}{2}|A_0<B_0)
\end{align*}

Goal: 
\begin{align*}
\EE[\text{times of choosing arm $x$ in $T$ rounds}]&=\Sigma_{i=1}^T{\PP(\text{choose arm $x$ at round }i)}\\
&=\Sigma_{X_1X_2...X_T}{\PP(X_1X_2...X_T)\cdot(\text{number of }X=B)}
\end{align*}

Proof:
\begin{align*}
\PP(\text{choose B at round }i)&=\Sigma_{X_1..X_i..X_T}{\PP(X_1..X_i..X_T)} \text{ where } X_i=B
\end{align*}

\end{document}


